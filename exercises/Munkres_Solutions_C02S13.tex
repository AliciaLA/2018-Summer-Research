\documentclass[12pt]{article}
\usepackage{latexsym, amssymb, amsmath, amsfonts, amscd, amsthm}
\usepackage{enumerate}
\usepackage[margin=1in]{geometry}
\linespread{1} %Change the line spacing only if instructed to do so.

\newenvironment{problem}[2][Problem]
{
	\begin{trivlist} 
		\item[\hskip \labelsep {\bfseries #1 #2:}]
	}
{
	\end{trivlist}
	}

\newenvironment{solution}[1][Solution]
{
	\begin{trivlist} 
		\item[\hskip \labelsep {\itshape #1:}]
	}
	{
	\end{trivlist}
}

\newenvironment{collaborators}[1][Collaborator(s)]
{
	\begin{trivlist} 
		\item[\hskip \labelsep {\bfseries #1:}]
	}
	{
	\end{trivlist}
}

%%%%%%%%%%%%%%%%%%%%%%%%%%%%%%%%%%%%%%%%%%%%%%%%%%
%%%%%%%%%%%%%%%%%%%%%%%%%%%%%%%%%%%%%%%%%%%%%%%%%%
%%%%%%%%%%%%%%%%%%%%%%%%%%%%%%%%%%%%%%%%%%%%%%%%%%
%
%
%    You need only modify code below this block.
%
%
%%%%%%%%%%%%%%%%%%%%%%%%%%%%%%%%%%%%%%%%%%%%%%%%%%
%%%%%%%%%%%%%%%%%%%%%%%%%%%%%%%%%%%%%%%%%%%%%%%%%%
%%%%%%%%%%%%%%%%%%%%%%%%%%%%%%%%%%%%%%%%%%%%%%%%%%
%
%
%%%%%%%%%%%%%%%
%
% Modify the title, author, and due date for each assignment:
%
%%%%%%%%%%%%%%%
\title{Munkres Chapter 2.13 exercises}
\author{Cherie Li}
\date{05/28/2018}

%%%%%%%%%%%%%%%
%
% Do not modify:
%
%%%%%%%%%%%%%%%
\begin{document}
\maketitle
\newpage
%%%%%%%%%%%%%%%
%
% Your problem statements and solutions start here.
% Use the \newpage command between problems so that
% each of your problems begins on its own page.
%
%%%%%%%%%%%%%%%
%Provide the problem statement.
%
%
\begin{problem}[Exercise]{2.13.4}
	\hfill	
	\begin{enumerate}[(a)]
		\item If $\lbrace{T_\alpha \rbrace}$ is a family of topologies on $X$ show that $\cap T_\alpha$ is a topology on $X$. Is $\cup T_\alpha$ a topology on $X$? 
		\item Let $\lbrace{T_\alpha\rbrace}$ be a family of topologies on $X$. Show that there is a unique smallest topology on $X$ containing all the collections $T_\alpha$, and a unique largest topology contained in all $T_\alpha$.  
		\item If $X = \lbrace{a, b, c\rbrace}$, let $T_1 = \lbrace{\varnothing, X, \lbrace{a\rbrace},  \lbrace{a,b\rbrace}\rbrace}$ and $T_2 = \lbrace{{a\rbrace},  \lbrace{b,c\rbrace}\rbrace}$ Find the smallest topology containing $T_1$ and $T_2$ and the largest topology contained in $T_1$ and $T_2$. 
	\end{enumerate}
\end{problem}
\begin{solution}
	\hfill
	\begin{enumerate}[(a)]
		\item To show that $\cap T_\alpha$ is a topology on $X$ we need to show that the three properties of topology hold. Note that $X, \varnothing \in \cap T_\alpha$ since each $T_\alpha$ is a topology, and by property of topology, $X, \varnothing \in T_\alpha$. Now we want to show that $\cap T_\alpha$ is closed under arbitrary unions. Notice that for all $T_\alpha$, $\cap T_\alpha \subseteq T_\alpha$.  Since each $T_\alpha$ is a topology, it is closed under arbitrary unions, so $\cap T_\alpha$ is closed under arbitrary unions. Next, we show that $\cap T_\alpha$ is closed under finite intersections. Notice again that $\cap T_\alpha \subseteq T_\alpha$ for each $T_\alpha$. Since each $T_\alpha$ is a topology, it also contains the finite intersection of any elements. Thus, finite intersections of $\cap T_\alpha$ are contained in all $T_\alpha$, so are contained in $\cap T_\alpha$. 
		\item Let $T$ be the topology generated by the subbasis $\cup T_\alpha$ where $\lbrace{T_\alpha\rbrace}$ are all the topologies of $X$. By definition of subbasis, $T$ is the collection of all unions of finite intersections of elements of $\cup T_\alpha$, so $T_\alpha \subseteq T$ for all $T_\alpha$. is an element  Suppose that there is another topology $T_1$ that contains all topologies on $X$. Then $T \subseteq T_1$, since by definition of topology, all unions of elements and finite intersections must be open sets as well. So $T$ is the unique smallest topology on $X$. \\
		For the unique largest topology on $X$ contained in all $T_\alpha$, consider the intersection of all topologies $\cap T_\alpha$. Suppose that there is another topology $T_2$ that is contained in all of $T_\alpha$. Then $T_2 \in \cap T_\alpha$, so this is the unique largest topology. 
		\item Taking $T_1 \cup T_2$ as our subbasis should give us the largest topology. The topology generated by unions of finite intersections of $T_1 \cup T_2$ is $T_3 = \lbrace{\varnothing, X , \lbrace{a\rbrace}, \lbrace{b\rbrace}, \lbrace{a,b\rbrace},\lbrace{b,c\rbrace}\rbrace}$. The intersection $T_1 \cap T_2$ should give the smallest topology contained in them, which is $T_4 = \lbrace{\varnothing, X, \lbrace{a\rbrace}\rbrace}$
		
	\end{enumerate}
	
\end{solution}
\begin{problem}[Exercise]{2.13.5}
	Show that if $A$ is a basis for a topology on $X$, then the topology generated by $A$ equals the intersection of all topologies on $X$ that contain $A$. Prove the same if $A$ is a subbasis. 
\end{problem}
\begin{solution}
	Let $X$ be a set together with a topology $T$. Let $A$ be a basis for $T$. We want to show that $T = \cap T_{\alpha}$, where $\lbrace{T_\alpha\rbrace} := \lbrace{T_\alpha | A \subseteq T_\alpha \rbrace}$. We show by double containment. First, we show that $T \subseteq \cap T_{\alpha}$. By Lemma 13.1, $T$ is the collection of all unions of elements of $A$. Since $A \in \cap T_{\alpha}$, and by our previous results $\cap T_{\alpha}$ is a topology on $X$, it is closed under arbitrary unions, so $T \in \cap T_{\alpha}$. Now, we show that $\cap T_\alpha \in T$. Notice that $A \subseteq T$, so $T = T\alpha$ for some $\alpha$. Therefore, $\cap T_\alpha \in T$. 
\end{solution}
\begin{problem}[Exercise]{2.13.6}
Show that the topologies of $\mathbb{R}_l$ and $\mathbb{R}_k$ are not comparable. 
\end{problem}
\begin{solution}
	It suffices to show that neither is $\mathbb{R}_l$ is not finer than $\mathbb{R}_k$ nor is $\mathbb{R}_k$ finer than $\mathbb{R}_l$. By definition of finer, we want to show that $\mathbb{R}_k \not\subset \mathbb{R}_l$ and $\mathbb{R}_l \not\subset \mathbb{R}_k$ respectively. First we show that  $\mathbb{R}_k \not\subset \mathbb{R}_l$. Let $B_l$ be a basis for $\mathbb{R}_l$. Let the set $B \in B_l$ be a basis element. Consider the basis element $\left[a,b\right) \in B$. Notice that there is no open interval of the form $(c,d)$ or $(c,d)- K$ such that $(c,d) \subseteq \left[a,b\right)$ and $a \in (c,d)$. Note that if $(c,d) \subseteq \left[a, b\right)$, then $a < c$. If $a \in (c,d)$, then $a > c$. By Lemma 13.3, $\mathbb{R}_l$ is not finer than $\mathbb{R}_k$. Now we want to show that $\mathbb{R}_l \not\subset \mathbb{R}_k$. Let $B_k$ be a basis for $\mathbb{R}_k$. Let $B' \subset B_k$. Consider the basis element $(-1, 1) - K \in B'$. Notice that there is no half open interval $\left[a,b\right)$ such that $0 \in (a,b)$ and $(a,b) \subset (-1,1)- K$. If $0 \in (a,b)$, then $b > 0$, and there exists $n\in\mathbb{Z}^+$ such that $0 < \frac{1}{n} < b$. So $(a,b) \not\subset (-1,1) - K$. If $\left[a,b\right) \subset (-1,1) -K$, then there does not exist $\frac{1}{n}$ for $n \in \mathbb{Z}^+$ such that $\frac{1}{n} \in \left[a, b\right)$. Then $0 \notin \left[a, b\right)$. 
\end{solution}
\end{document}
