\documentclass[12pt]{article}
\usepackage[margin=1in]{geometry}

\usepackage{latexsym, amssymb, amsmath, amsfonts, amscd, amsthm}
\usepackage{enumerate, hyperref, multicol, tikz}

\theoremstyle{definition}
\newtheorem{definition}{Definition}
\newtheorem{example}{Example}
\newtheorem{exercise}{Exercise}
\newtheorem{question}{Question}
\newtheorem*{notation*}{Notation}

\newtheorem{fact}{Fact}
\newtheorem{lemma}{Lemma}
\newtheorem{proposition}{Proposition}
\newtheorem{theorem}{Theorem}
\newtheorem{corollary}{Corollary}

\newcommand{\NN}{\mathbb{N}}
\newcommand{\ZZ}{\mathbb{Z}}
\newcommand{\QQ}{\mathbb{Q}}
\newcommand{\RR}{\mathbb{R}}
\newcommand{\CC}{\mathbb{C}}
\newcommand{\FF}{\mathbb{F}}

\usepackage{amsfonts}
\usepackage{enumerate}

\begin{document}
\bfseries{Alicia Ledesma Alonso}

\begin{exercise} Munkres Chapter 2, section 13, problem 4: 
\begin{enumerate}[(a)]
	\item \begin{proof}
		To show that $\cap{\mathfrak{T}_\alpha}$ is topology on $X$, we show that it obeys the three conditions of a topology. Now since $\{\mathfrak{T}_\alpha\}$ is an indexed family of topologies, for all $\alpha$, both the empty set $\{\}$ and $X$ are in $\mathfrak{T}_\alpha$. Because of this, for all $\alpha$, we have that $\mathfrak{T}_\alpha \in\cap\mathfrak{T}_\alpha$ by definition of intersection, and so $\{\}$ and $X$ are in $\cap\mathfrak{T}_\alpha$. Let $S_1,S_2\subset\cap{\mathfrak{T}_\alpha}$ be arbitrary subcollections. Now, again by definition of intersection, we have that $S_1,S_2\subset\mathfrak{T}_\alpha$. Since $\{\mathfrak{T}_\alpha\}$ is an indexed family of topologies, we have that $S_1\cup S_2 \in \mathfrak{T}_\alpha$. By the same reasoning above, $S_1\cup S_2\in \cap\mathfrak{T}_\alpha$. Since $S_1,S_2\in \cap\mathfrak{T}_\alpha$ were arbitrary, the union of the elements of any subcollection of $\cap\mathfrak{T}_\alpha$ is in $\cap\mathfrak{T}_\alpha$. Similary, if we take an arbitrary finite subcollection $S_1,S_2\subset\cap{\mathfrak{T}_\alpha}$, since $\{\mathfrak{T}_\alpha\}$ is an indexed family of topologies, $S_1\cap S_2\in \mathfrak{T}_\alpha$ by definition of intersection and so $S_1\cap S_2\in \cap\mathfrak{T}_\alpha$. So, the intersection of any finite subcollection of $\cap\mathfrak{T}_\alpha$ is in $\cap\mathfrak{T}_\alpha$. Thus, $\cap\mathfrak{T}_\alpha$ is a topology on $X$. 
	\end{proof}
	However, the unions of any arbitrary subcollections of $\cup\mathfrak{T}_\alpha$ does not imply the union of those arbitrary subcollections in $\mathfrak{T}_\alpha$ because unions of subcollections do not necessarily preserve conditions necessary to define a topology. Thus, $\cup\mathfrak{T}_\alpha$ is not a topology on $X$.
	\item The unique smallest topology on $X$ can be found by taking the intersection of all
	    the subcollections that form the discrete topology on $\cup\mathfrak{T}_\alpha$. We look in $\cup\mathfrak{T}_\alpha$ because the union of the indexed family $\mathfrak{T}_\alpha$ will be smaller than the intersection $\cap\mathfrak{T}_\alpha$. The unique largest topology on X is found in the intersection $\cap\mathfrak{T}_\alpha$ because as we have discussed earlier, since by definition of intersection, it is the intersection that preserve properties of a topology and thus give a larger topology. 
	\item The smallest topology containing both $\mathfrak{T}_1$ and $\mathfrak{T}_2$ is
	    found by taking the union, which is \begin{equation}\{\{\},X,{a},{b},{a,b},{b,c}\}.\end{equation} \newline The largest topology containing both $\mathfrak{T}_1$ and $\mathfrak{T}_2$ is found by taking the intersection, which is \begin{equation}\{\{\},X,{a}\}.\end{equation}
\end{enumerate}
\end{exercise}

\begin{exercise} Munkres Chapter 2, section 13, problem 5:
\begin{proof}
	Suppose $\mathfrak{A}$ is a basis for a topology on $X$. We show that the topology generated by $\mathfrak{A}$ equals the intersection of all topologies on X that contain $\mathfrak{A}$. By the definition of a topology generated by $\mathfrak{A}$, if $\mathfrak{a}$ is the topology generated by $\mathfrak{A}$, then $\mathfrak{a}\subset \cap\{\mathfrak{T}_\alpha\mid\mathfrak{A}\subset \mathfrak{T}_\alpha\}$ by default since $\mathfrak{a}\subset \mathfrak{A}$. Our job is to show $\cap\{\mathfrak{T}_\alpha\mid\mathfrak{A}\subset \mathfrak{T}_\alpha\}\subset \mathfrak{a}$. Let $S\subset \cap\{\mathfrak{T}_\alpha\mid\mathfrak{A}\subset \mathfrak{T}_\alpha\}$ be an arbitrary open set for an arbitrary $\alpha$. By Lemma 13.1, since $\mathfrak{A}$ is a basis for $\mathfrak{T}$ then $\mathfrak{T}\subset \cap\mathfrak{A}$, and so there must exist an $\alpha$ such that $S\subset\mathfrak{A}\subset\mathfrak{T}$. Since $\mathfrak{a}\subset\mathfrak{A}$ then $S\subset\mathfrak{a}$ by definition of intersection. Thus, $\cap\{\mathfrak{T}_\alpha\mid\mathfrak{A}\subset \mathfrak{T}_\alpha\}\subset \mathfrak{a}$.
	If $\mathfrak{A}$ is a subbasis, then by definition, $\cap\mathfrak{A}=X$. Let $S\subset\cap\mathfrak{A}$ be arbitrary. By definition of intersection, there exists an $\alpha$ such that $S\subset\cap\{\mathfrak{T}_\alpha\mid\mathfrak{A}\subset \mathfrak{T}_\alpha\}$. By definition of a topology, since $\mathfrak{T}$ is a topology on X, then there exists an $\alpha$ such that $S\subset\mathfrak{T}$. Since $\mathfrak{T}$ contains $\mathfrak{A}$ and its elements then, by definition of intersection, $S\subset\mathfrak{a}$.
\end{proof}
\end{exercise}

\begin{exercise} Munkres Chapter 2, section 13, problem 6:
\begin{proof}
	We show that $\mathbb{R}_l$ and $\mathbb{R}_k$ are not comparable.Take an arbitrary basis element of $\mathbb{R}_l$, $[a,b)$, where $a,b\in\mathbb{R}$ are fixed. Because this is a half-open interval, there is no open interval that contains a and lies in $[a,b)$. Similary, if we take an arbitrary basis element of $\mathbb{R}_k$, $(-1,1)-K$, where $K$ is a fixed rational, then there does not exist a half-open interval that contains 0 and lies in $(-1,1)-K$.
\end{proof}
\end{exercise} 

\begin{exercise} Munkres Chapter 2, section 16, problem 1:
\begin{proof}
	We show that if $Y$ is a subspace of $X$, and $A$ is a subset of $Y$, then the topology $A$ inherits as a subspace of $Y$ is the same as the topology it inherits as a subspace of $X$. By definition of a subspace, since $Y$ is a subspace of $X$, then $\mathfrak{T}_Y = \{Y\cap U\mid U\in \mathfrak{T}\}$ where $(X,\mathfrak{T})$ is a topology. Now, we are told that $A\subset Y$ so we can write $\mathfrak{T}_A = \{A\cap U\mid U\in \mathfrak{T}_Y\}$ as the topology $A$ as a subspace of Y. We want to show that $\{A\cap U\mid U\in \mathfrak{T}_Y\} = \{A\cap U\mid U\in \mathfrak{T}\}$ where the right hand side denotes the topology $A$ as a subspace of $X$. By Lemma 16.2, since $U$ is open, then $\{A\cap U\mid U\in \mathfrak{T}\}$ contains open sets since $\mathfrak{T}_Y\subset \mathfrak{T}$ and thus, $\{Y\cap U\mid U\in \mathfrak{T}_Y\}=\{Y\cap U\mid U\in \mathfrak{T}\}$.
\end{proof}
\end{exercise}

\begin{exercise} Munkres Chapter 2, section 16, problem 4
\begin{proof}
	We show that $\pi_1: X\times Y\to X$ and $\pi_2: X\times Y\to Y$ are open maps. Let $S\subset X$ and $J\subset Y$ open sets. By the definition of a product topology, there exists a basis for $X\times Y$ in the form of $S\times J$. Let $U\subset X\times Y$. Let $s\subset\pi_1(U)$. Since $U$ is open, let $s\times j\in U$. We know by definition that $\pi_1(S\times J)=S$ which is a subset of $\pi_1(U)$ because $U\subset S\times J$. Thus, $\pi_1(U)$ is open. A similar approach is done for $\pi_2(U)$.
	 
\end{proof}
\end{exercise}

\begin{exercise} Munkres Chapter 2, section 16, problem 5\newline Let $X$ and $X'$ denote a
	single set in the topologies $\mathfrak{T}$ and $\mathfrak{T}'$, respectively. Let $Y$ and $Y'$ denote a single set in the topologies $\mathfrak{U}$ and $\mathfrak{U}'$, respectively. Assume these sets are nonempty. 
\begin{enumerate}[(a)]
	\item \begin{proof}
		If $\mathfrak{T}'\supset \mathfrak{T}$ and $\mathfrak{U}'\supset \mathfrak{U}$, then we can write $X\times Y\subset X'\times Y'$.' By definition of finer, we have that $X'\times Y$ is finer than $X\times Y$.
	\end{proof}
	\item \begin{proof}
		The converse holds true. Let $W\subset X$ and $x\in X$ be open, and let $V\subset Y$ and $y\in Y$ be open. Then, $W\times V$ is open in $X\times Y$ and since $X'\times Y'$ is finer, then $X\times Y\subset X'\times Y'$. For the product topology $X'\times Y'$ there must exist a basis $A\times B$ that contains $x\times y$ and is a subset of $W\times V$. With this, we can say that there are open sets in the basis $A\times B$ that is also open in $X'\times Y'$. So $\mathfrak{T}'$ and $\mathfrak{U}'$ are finer than $\mathfrak{T}$ and $\mathfrak{U}$ respectively.
	\end{proof}
\end{enumerate} 
\end{exercise}

\begin{exercise} Munkres Chapter 2, section 16, problem 10 \newline Let $I=[0,1]$. We
    compare the topologies by studying their basis elements. In the product topology on $I\times I$, we can take an element such as $(0.2,0.5)\times(0.3,0.4)$. This is open in the product topology but it is not open in the dictionary order topology on $I\times I$. Elements of of the dictionary order topology are different such as $\{0.1\}\times (0.2,1)$. These two topologies are not comparable. If we look at the topology $I\times I$ inherits as a subspace of $\mathbb{R}\times \mathbb{R}$ in the dictionary order topology, then we find that this topology is finer than the other two listed above. This is because the first two topologies can be constructed by taking unions of the elements found in the third topology. That is to say, the product topology is finer than the dictionary order topology, which itself is finer than the subspace inherited by the dictionary topology. 
\end{exercise}

\begin{exercise} Munkres Chapter 2, section 17, problem 1 (accidentally did this one)
\begin{proof}
	Let $\mathfrak{C}$ be a collection of subsets of the set $X$. Suppose that $\{\}$ and $X$ are in $\mathfrak{C}$, and that finite unions and arbitrary intersections of elements of $\mathfrak{C}$ are in $\mathfrak{C}$. We show that the collection $\mathfrak{T}=\{X-C\mid C\in\mathfrak{C}\}$ is a topology on $X$. We use the definition of a topology. It is given that $\{\}$ and $X$ are in $\mathfrak{T}$ since $\{\}$ and $X$ are in $\mathfrak{C}$. Let $U,V\in\mathfrak{T}$ be arbitrary and define as $U=X-C_U$ and $V=X-C_V$. Note that $U\cup V=(X-C_U)\cup(X-C_V)=X-(C_U\cap C_V)$ by DeMorgan's Law. Since $C_U\cap C_V\in\mathfrak{C}$, then $U\cup V\in\mathfrak{T}$ and so the union of the elements of any subcollection of $\mathfrak{T}$ are in $\mathfrak{T}$. Note also that $U\cap V=(X-C_U)\cap(X-C_V)=X-(C_U\cup C_V)$ by DeMorgan's Law. Since $C_U\cup C_V\in\mathfrak{C}$, then $U\cap V\in\mathfrak{T}$ and so the intersection of the elements of any subcollection of $\mathfrak{T}$ are in $\mathfrak{T}$. Thus, $\mathfrak{T}$ is a topology since it satisfies the three conditions of a topology.
\end{proof}
\end{exercise}

\begin{exercise} Munkres Chapter 2, section 17, problem 4
\begin{proof}
	We show that if $U$ is open in $X$ and $A$ is closed in $X$, then $U-A$ is open in $X$, and $A-U$ is closed in $X$. By definition, $U-A=U\cap (X-A)$ and $A-U=A\cap (X-U)$. Now since $X$ is both open and closed, then $X-U$ is closed and $X-A$ is open. Thus, $A\cap (X-U)$ is closed and $U\cap (X-A)$ is open.
\end{proof}
\end{exercise}

\begin{exercise} Munkres Chapter 2, section 17, problem 11  
\begin{proof}
	We show that the product of two Hausdorff spaces is Hausdorff. Let $V$ and $W$ be two Hausdorff spaces, and take elements from the product, $a\times b, c\times d\in V\times W$. Assume that $a\neq b$ and $c\neq d$. Because $V$ is a Hausdorff space, then there exist a neighborhood $U_a$ for the point $a$ and $U_c$ for the point $c$ such that $U_a\cap U_c=0$. Then, we will have $U_a\times W$ be a neighborhood for $a\times b$ and $U_c\times V$ be a neighborhood for $c\times d$ because $a,c\in V$ and $b,d\in W$. Now since $U_a\cap U_c=0$, then $U_a\times W$ and $U_c\times V$ are disjoint for $a\times b$ and $c\times d$ respectively. Thus, $V\times W$ is Hausdorff.  
\end{proof}
\end{exercise}

\begin{exercise} Munkres Chapter 2, section 17, problem 12 
\begin{proof}
	We show that a subspace of a Hausdorff space is Hausdorff. Let $X$ be a Hausdorff space with a topology $\mathfrak{T}$. If $Y$ is subset of $X$, the collection $\mathfrak{T}_Y=\{Y\cap U\mid U\in\mathfrak{T}\}$ is a topology on $Y$ called the subspace topology where $Y$ is a subspace of the Hausdorff space $X$. By definition, $\mathfrak{T}$ contains open sets that consist of all intersections of open sets of $X$ with $Y$. By definition of Hausdorff, an open set of $X$ will contain distinct points with respective open neighborhoods that are disjoint. Say we take arbitrary points $x_1,x_2\in\mathfrak{T}$ such that $x_1\neq x_2$ and $U_1$ and $U_2$ are their respective open neighborhoods for $x_1$ and $x_2$. Now since $Y$ is subset of $X$ then $Y$ consists of open sets from $X$ such that each point in $Y$ is unique and has disjoint neighborhoods. Because $\mathfrak{T}$ consists of open sets of intersections between $Y$ and $X$, then $Y$ consists of the points $x_1$ and $x_2$ and their respective disjoint neighborhoods $U_1$ and $U_2$. Since these points were arbitrary, it follows that $Y$ is a Hausdorff by definition. 
\end{proof}
\end{exercise}

\begin{exercise} Munkres Chapter 2, section 17, problem 13 
\begin{proof}
	We show that $X$ is Hausdorff if and only if the diagonal $\Delta=\{x\times x\mid x\in X\}$ is closed in $X\times X$. First we assume that $\Delta$ is closed. We show that $X$ is a Hausdorff space. Because $\Delta$ is closed, then its complement is open, i.e. $X\times X/\Delta$ is open. Let $p\in X\times X/\Delta$ be an arbitrary point defined as $p=(x_1,x_2)$ where $x_1,x_2\in X$ and $x_1\neq x_2$. Since $p$ is open, then there exists a subset $Y$ that is open such that $p\in Y\subset X\times X/\Delta$. Define $Y=U_1\times U_2$ and since $Y$ is open, then $U_1$ and $U_2$ are open neighborhoods in $X$ for the points $x_1$ and $x_2$ respectively. Now since $x_1\neq x_2$, then $U_1\cap U_2=\{\}$. Thus, $X$ is a Hausdorff space. Now we assume that $X$ is a Hausdorff space, and we show that $\Delta=\{x\times x\mid x\in X\}$ is closed in $X\times X$, i.e. $X\times X/\Delta$ is open. Let $p\in X\times X/\Delta$ be arbitrary and define as $p=(x_1,x_2)$ where $x_1,x_2\in X$. Since $X$ is Hausdorff then, $x_1\neq x_2$ and there exists open neighborhoods, $U_1$ for $x_1$ and $U_2$ for $x_2$ such that $U_1\cap U_2=\{\}$. From a previous problem, the product of two Hausdorff spaces is a Hausdorff space, and so $X\times X$ is Hausdorff. Because of this, $U_1\times U_2$ is also an open neighborhood for $p$. Since $p\in X\times X/\Delta$ was arbitrary and open, then $X\times X/\Delta$ is open, and $\Delta$ is closed. 
\end{proof}
\end{exercise}

\begin{exercise} Munkres Chapter 2, section 18, problem 5
\begin{proof}
	We show that the subspace $(a,b)$ of $\mathbb{R}$ is homeomorphic with $(0,1)$ and the subspace $[a,b]$ of $\mathbb{R}$ is homeomorphic with $[0,1]$. In order to show that $(a,b)$ is homeomorphic to $(0,1)$ we construct a function $f$ such that $f:(a,b)\to (0,1)$ is bijective and its inverse is continuous. One way to do this is to let $f(a)=0$ and $f(b)=0$. We can then form a line given the slope $\frac{f(b)-f(a)}{b-a}$ and choose the intercept to be $\frac{-a}{b-a}$. Then we can use the slope intercept formula to construct the function to be $f(x)=\frac{1}{b-a}x+\frac{-a}{b-a}$ where $a<x<b$. Similarly, in order to show that the subspace $[a,b]$ of $\mathbb{R}$ is homeomorphic with $[0,1]$, we only need to modify our previous response to include the end points. Thus, a function for this homeomorphism would be $f(x)=\frac{1}{b-a}x+\frac{-a}{b-a}$ but with $a\leq x\leq b$.
\end{proof}
\end{exercise}

\begin{exercise} Munkres Chapter 2, section 18, problem 8 \newline
	Let $Y$ be an ordered set in the order topology. Let $f,g:X\to Y$ be continuous.
\begin{enumerate}[(a)]
	\item \begin{proof}
		We show that the set $\{x\mid f(x)\leq g(x)\}$ is closed in $X$. Say $M=\{x\mid f(x)\leq g(x)\}$, then we show that the complement is open, i.e. we show that $X-M$ is open. Suppose that $f(x)=g(x)$ then we have that $X-M=\{\}$ since $X$ contains all of $f(x)$ (or $g(x)$). Since the empty set is always open, then $X-M$ is open here. Now suppose that $f(x)\neq g(x)$. Then, $X-M=\{x\mid f(x)> g(x)\}$. We show that this is open by first realizing that the order topology is Hausdorff, and so that means that $Y$ is also Hausdorff. With this, and since the functions are continuous, then we can define open neighborhoods in $Y$ for each function so that the open neighborhoods are disjoint. Let $U_f\in X$ be an open neighborhood for $f(x)$ and let $U_g\in X$ be an open neighborhood for $g(x)$. In the complement, we have that for any point $x_f\in U_f$ and $x_g\in U_g$ then $f(x_f)>g(x_g)$. Since $f$ and $g$ are continuous, then their respective inverses exist and are open in $X$. We can define $U=\{U_f,U_g\in X\mid f^-1(U_f)\cap g^-1(U_g)\}$ that is contained in the complement $X-M$ to be an open set. Note that for any element in $U$, there will be a respective function defined in $X$ since the functions are continuous. Thus, $X-M$ is an open set since all its components in the set will be open.
		\end{proof} 
	\item \begin{proof}
		Let $h:X\to Y$ be the function $h(x)=min\{f(x),g(x)\}$. We show that $h$ is continuous. We can define $h$ to be the following piece-wise function: \[h=\begin{cases}f(x)&f(x)\leq g(x)\\g(x)&g(x)\leq f(x).\end{cases}\] From part (a), we know that each set of $h$ is closed. By the pasting lemma since these sets are closed and $f$ and $g$ are continuous, then $h$ is a continuous function. 
	\end{proof}
\end{enumerate}
\end{exercise}

\begin{exercise} Munkres Chapter 2, section 20, problem 1 
\begin{enumerate}[(a)]
	\item In $\mathbb{R}^n$, define \[d'(\mathbf{x},\mathbf{y})=\mid x_1-y_1\mid +...+\mid x_n-y_n\mid.\] 
	\begin{proof}
		We show that $d'$ is a metric that induces the usual topology of $\mathbb{R}^n$. First we show that $d'$ is a metric. By definition of a metric, we have that $d'$ must satisfy the three conditions. For the first condition, note that since $d'$ is defined by taking the absolute value of differences between 2-tuples of numbers, then, $d'(x,y)\geq 0$ for all $x,y\in X$. Moreover, the equality holds true for $x=y$. For the second condition, note again by the definition of taking the absolute value of 2-tuples of numbers that $d'(x,y)=d'(y,x)$ for all $x,y\in X$. For the third condition, we will use the following inequality: \[|a+b|\leq |a|+|b|.\] We start off by writing out our triangle inequality: \[|x_1-y_1|+...+|x_n-y_n|+|y_1-z_1|+...+|y_n-z_n|\geq |x_1-z_1|+...+|x_n-z_n|.\] By applying the inequality above, we can cancel the $y_1...y_n$ terms on the left side of the inequality. Say we look at the first dimension. We start with using the first inequality listed and substituting our first dimensional terms: $|x_1-y_1|+|y_1-z_1|\geq |x_1-y_1+y_1-z_1|=|x_1-z_1|$ and so $|x_1-y_1|+|y_1-z_1|\geq |x_1-z_1|$. Expanding this to n-dimensions works as well since this inequality works in $\mathbb{R}^n$. \newline Now that we have shown that $d'$ is a metric, we show that it induces the usual topology of $\mathbb{R}^n$ which has the Euclidean metric $d$ defined as \[d(\mathbf{x},\mathbf{y})=||\mathbf{x}-\mathbf{y}||=\sqrt{(x_1-y_1)^2+...+(x_n-y_n)^2}.\] Note that we can rewrite the Euclidean metric as follows: \[d(\mathbf{x},\mathbf{y})=\sqrt{\sum_{i=1}^{n}(x_i-y_i)^2}\]
		\[d(\mathbf{x},\mathbf{y})\leq\sqrt{n}\,\sum_{i=1}^{n}\sqrt{(x_i-y_i)^2}\]
		\[d(\mathbf{x},\mathbf{y})\leq\sqrt{n}\,\sum_{i=1}^{n}|x_i-y_i|\]
		\[d(\mathbf{x},\mathbf{y})\leq\sqrt{n}\,d'(\mathbf{x},\mathbf{y}).\]
		Then, by Lemma 20.2, we can find a new parameter for the basis of $d'$ such that \[B_{d'}(\mathbf{x},\epsilon/\sqrt{n})\subset B_{d}(\mathbf{x},\epsilon)\]
		Note also that $\sqrt{a}\leq a$ for any $a\in\mathbb{R}^{+}$. Then, $d(\mathbf{x},\mathbf{y})\leq d'(\mathbf{x},\mathbf{y})$ for all $\mathbf{x},\mathbf{y}\in\mathbb{R}^{n}$. So we can write \[B_{d}(\mathbf{x},\epsilon)\subset B_{d'}(\mathbf{x},\epsilon)\] 
		Since we were able to find basis elements of each metric that are in each other, this shows that $d'$ induces the usual topology of $\mathbb{R}^{n}$.
		\end{proof}
	Now, we sketch the basis elements under $d'$ when $n=2$. The basis elements have a radius of $\epsilon/2$. However, unlike the euclidean metric, the $d'$ metric is defined in a way such that distance is measured as if walking around the edges of little squares to get from one point to another. Thus, the basis elements of $d'$ are squares.
	\item \[d'(\mathbf{x},\mathbf{y})=\Bigg[\sum_{i=1}^{n}|x_i-y_i|^p\Bigg]^{1/p}\] 
	\[d'(\mathbf{x},\mathbf{y})\leq\sqrt{n}\,\sum_{i=1}^{n}\big[|x_i-y_i|^p\big]^{1/p}\]
	\[d'(\mathbf{x},\mathbf{y})\leq\sqrt{n}\,\sum_{i=1}^{n}|x_i-y_i|\]
	Let $m(\mathbf{x},\mathbf{y})=\sum_{i=1}^{n}|x_i-y_i|$ be the taxi cab metric. From part (a) we saw that this induced the usual topology on $\mathbb{R}^{n}$.  
	By Lemma 20.2, we can find a new parameter for the basis of $d'$ such that \[B_{m}(\mathbf{x},\epsilon/\sqrt{n})\subset B_{d'}(\mathbf{x},\epsilon)\]
	and \[B_{d'}(\mathbf{x},\epsilon)\subset B_{m}(\mathbf{x},\epsilon).\] 
	We are able to find basis elements of each metric that are in each other. We know that $d'$ is contained and contains the taxi cab metric as shown above. This shows that $d'$ induces the taxi cab metric which we know from part (a) to induce the topology of $\mathbb{R}^{n}$.
\end{enumerate}
\end{exercise}

\begin{exercise} Munkres Chapter 2, section 20, problem 2 
\begin{proof}
	We show that $\mathbb{R}\times\mathbb{R}$ in the dictionary order topology is metrizable. 
\end{proof}
\end{exercise}

\begin{exercise} Munkres Chapter 2, section 20, problem 3 \newline
	Let $X$ be a metric space with metric $d$. 
\begin{enumerate}[(a)]
	\item \begin{proof}
		We show that $d:X\times X\to\mathbb{R}$ is continuous. 
	\end{proof}
	\item \begin{proof}
		Let $X'$ denote a space having the same underlying set as $X$. We show that if $d:X'\times X'\to\mathbb{R}$ is continuous, then the topology of $X'$ is finer than the topology of $X$. 
	\end{proof} 
\end{enumerate}
\end{exercise}

\begin{exercise} Munkres Chapter 2, section 20, problem 11
\begin{proof}
	We show that if $d$ is a metric for $X$, then \[d'(x,y)=\frac{d(x,y)}{1+d(x,y)}\] is a bounded metric that gives the topology of X. (hint if f(x)=x/(1+x) for x$>$0, use the mean value theorem to show that f(a+b)-f(b)$\leq$f(a). 
\end{proof}
\end{exercise}

\end{document}