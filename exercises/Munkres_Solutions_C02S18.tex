\documentclass[12pt]{article}
\usepackage{latexsym, amssymb, amsmath, amsfonts, amscd, amsthm}
\usepackage{enumerate}
\usepackage[margin=1in]{geometry}
\usepackage{calrsfs}
\DeclareMathAlphabet{\pazocal}{OMS}{zplm}{m}{n}
\newcommand{\La}{\mathcal{L}}
\newcommand{\Lb}{\pazocal{L}}
\linespread{1} %Change the line spacing only if instructed to do so.
\newenvironment{problem}[2][Problem]
{
	\begin{trivlist} 
		\item[\hskip \labelsep {\bfseries #1 #2:}]
	}
{
	\end{trivlist}
	}

\newenvironment{solution}[1][Solution]
{
	\begin{trivlist} 
		\item[\hskip \labelsep {\itshape #1:}]
	}
	{
	\end{trivlist}
}

\newenvironment{collaborators}[1][Collaborator(s)]
{
	\begin{trivlist} 
		\item[\hskip \labelsep {\bfseries #1:}]
	}
	{
	\end{trivlist}
}

%%%%%%%%%%%%%%%%%%%%%%%%%%%%%%%%%%%%%%%%%%%%%%%%%%
%%%%%%%%%%%%%%%%%%%%%%%%%%%%%%%%%%%%%%%%%%%%%%%%%%
%%%%%%%%%%%%%%%%%%%%%%%%%%%%%%%%%%%%%%%%%%%%%%%%%%
%
%
%    You need only modify code below this block.
%
%
%%%%%%%%%%%%%%%%%%%%%%%%%%%%%%%%%%%%%%%%%%%%%%%%%%
%%%%%%%%%%%%%%%%%%%%%%%%%%%%%%%%%%%%%%%%%%%%%%%%%%
%%%%%%%%%%%%%%%%%%%%%%%%%%%%%%%%%%%%%%%%%%%%%%%%%%
%
%
%%%%%%%%%%%%%%%
%
% Modify the title, author, and due date for each assignment:
%
%%%%%%%%%%%%%%%
\title{Munkres Chapter 2.18 exercises}
\author{Cherie Li}
\date{06/07/2018}

%%%%%%%%%%%%%%%
%
% Do not modify:
%
%%%%%%%%%%%%%%%
\begin{document}
\maketitle
\newpage
%%%%%%%%%%%%%%%
%
% Your problem statements and solutions start here.
% Use the \newpage command between problems so that
% each of your problems begins on its own page.
%
%%%%%%%%%%%%%%%
%Provide the problem statement.
%
\begin{problem}[Exercise]{2.18.5}
	Show that the subspace $(a, b)$ of $\mathbb{R}$ is homeomorphic with $(0,1)$ and the subspace $[a,b]$ of $\mathbb{R}$ is homeomorphic with $[0,1]$. 
\end{problem}
\begin{solution}
	Let $(a,b)$ be a subspace of $\mathbb{R}$. Consider the function $f(x) = \frac{x-a}{b-a}$, where $x \in (a,b)$. Notice that $0 < f(x) < 1$ and that it is bijective. Since $b > a$, $b-a > 0$, so the function is continuous, since it is just a linear function $x -a$ over some nonzero constant. Also, we have that the inverse is also continuous: $f(y) = y(b-a) + a$, as this is just the equation of a line with slope $(b-a)$ and intercept $a$ as well as bijective. \\
	Now consider the same function for $0 \leq f(x) \leq 1$. Notice when $x = a$, the function evaluates to $0$ which is in the interval $[0, 1]$. Also when $x = b$, we have that the function evaluates to 1. Also the function is continuous and bijective still. \\
	Since function and its inverse are both continuous, then $f$ is a homeomorphism and $(0,1)$ homeomorphic to $(a,b)$ and $[0,1]$ homeomorphic to $[a,b]$. 
\end{solution}
\begin{problem}[Exercise]{2.18.10}
	Let $Y$ be an ordered set in the order topology. Let $f, g: X \rightarrow Y$ be continuous. 
	\begin{enumerate}[(a)]
		\item Show that the set $\lbrace{x | f(x) \leq g(x)\rbrace}$ is closed in $X$. 
		\item Let $h: X \rightarrow Y$ be the function $h(x) = min\lbrace{f(x), g(x)\rbrace}$. Show that $h$ is continuous. [$Hint$: Use the pasting lemma.]
	\end{enumerate}
\end{problem}
\begin{solution} \hfill
	\begin{enumerate}[(a)]
	\item Let $Y$ be an ordered set in the order topology and let $f, g: X \rightarrow Y$ be continuous. To show that the set $A:= \lbrace{x | f(x) \leq g(x)\rbrace}$ is closed it suffices to show that $A^c :=  \lbrace{x | f(x) > g(x)\rbrace}$ is open in $X$. Assume $A^c$ is nonempty and let $a \in A^c$ be arbitrary. Let $y_1, y_2, y_3 \in Y$ such that $y_1 \leq g(a) < y_2 < f(a) \leq y_3$. Notice that $(y_1, y_2)$ and $(y_2, y_3)$ open in $Y$, since if $f(a)$ is the maximal element or $g(a)$ is the minimal element, we have $\left[g(a), y_2\right), \left(y_2, f(a)\right]$ instead. These are open intervals in order topology, so their intersection $(y_1, y_2) \cap (y_2, y_1)$ is also open. Since $f$ is continuous, by definition we have that $f^{-1}\big((y_1, y_2) \cap (y_2, y_1)\big)$ is also open in $X$ where $a \in f^{-1}\big((y_1, y_2) \cap (y_2, y_1)\big) \subset A^c$. So $A^c$ is open in $X$ and $A$ is closed.  
	\item Notice from the previous part that the set  $A:= \lbrace{x | f(x) \leq g(x)\rbrace}$ and symmetrically $B:= \lbrace{x | g(x) \leq f(x)\rbrace}$ are both closed in $X$. Also, notice that $X = A \cup B$ since all $x$ must have $g(x) > f(x)$, $g(x) = f(x)$ or $g(x) < f(x)$ and that their intersection is when $g(x) = f(x)$. By Theorem 18.3 (the pasting lemma), $f$ and $g$ give the continuous function $h: X \rightarrow Y$, where $h(x) = f(x)$ if $x \in A$. So if $x \in A$, then $f(x) < g(x)$. Similarly, if $x \in B$, then $g(x) > f(x)$. This is the same as the function min, defined in the statement. 
	\end{enumerate}
\end{solution}
\end{document}