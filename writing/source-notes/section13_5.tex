\documentclass[12pt]{article}
\usepackage{latexsym, amssymb, amsmath, amsfonts, amscd, amsthm}
\usepackage{enumerate,tikz}
\usetikzlibrary{arrows,automata,positioning}
\usepackage[margin=1in]{geometry}

\definecolor{darkblue}{rgb}{0, 0, .6}
\definecolor{grey}{rgb}{.7, .7, .7}

\linespread{1} %Change the line spacing only if instructed to do so.

\newenvironment{problem}[2][Problem]
{
	\begin{trivlist} 
		\item[\hskip \labelsep {\bfseries #1 #2:}]
	}
{
	\end{trivlist}
	}

\newenvironment{solution}[1][Solution]
{
	\begin{trivlist} 
		\item[\hskip \labelsep {\itshape #1:}]
	}
	{
	\end{trivlist}
}

\newenvironment{collaborators}[1][Collaborator(s)]
{
	\begin{trivlist} 
		\item[\hskip \labelsep {\bfseries #1:}]
	}
	{
	\end{trivlist}
}

\newtheorem{lemma}{Lemma}

%%%%%%%%%%%%%%%%%%%%%%%%%%%%%%%%%%%%%%%%%%%%%%%%%%
%%%%%%%%%%%%%%%%%%%%%%%%%%%%%%%%%%%%%%%%%%%%%%%%%%
%%%%%%%%%%%%%%%%%%%%%%%%%%%%%%%%%%%%%%%%%%%%%%%%%%
%
%
%    You need only modify code below this block.
%
%
%%%%%%%%%%%%%%%%%%%%%%%%%%%%%%%%%%%%%%%%%%%%%%%%%%
%%%%%%%%%%%%%%%%%%%%%%%%%%%%%%%%%%%%%%%%%%%%%%%%%%
%%%%%%%%%%%%%%%%%%%%%%%%%%%%%%%%%%%%%%%%%%%%%%%%%%
%%%%%%%%%%%%%%%
%
% Do not modify:
%
%%%%%%%%%%%%%%%
\begin{document}
%%%%%%%%%%%%%%%
%
% Your problem statements and solutions start here.
% Use the \newpage command between problems so that
% each of your problems begins on its own page.
%
%%%%%%%%%%%%%%%
%Provide the problem statement.

\begin{problem}{13.5} \ \\
Show that if \(A\) is a basis for a topology on \(X\), then the topology generated by \(A\) equals the intersection of all topologies on \(X\) that contain \(A\). Prove the same if \(A\) is a subbasis.
\end{problem}

\begin{proof}
Let \(T\) be the topology on \(X\) generated by \(A\). Let \(S \in T\) be arbitrary. Let \(\Gamma = \{T_\alpha\}\), the set of all topologies on \(X\) containing \(A\). Then, for all \(t \in \cap_{i \in \Gamma} i\), \(A\) is contained in \(t\) by the previous statement.  From the proof of Lemma 13.1, it follow that \(S\) is a union of elements in \(A\), i.e. \(S = \cup_{a \in A} a\). By the definition of basis, \(A \subseteq X\), and \(X \in \cap T_\alpha\), where \(\cap T_\alpha\) is the intersection of all toplogies on \(X\) that contain \(A\), by the definition of topology on \(X\) as the intersection is a topology on \(X\) by Exercise 13.4. Hence, it follows that \(A \in \cap T_\alpha\), and since \(S \subseteq A\) as S is the union of elements from \(A\), it follows that \(S \in \cap T_\alpha\) and hence \(T \subseteq \cap T_\alpha\). \newline \\
Since the topology \(T\) on \(X\) is generated by \(A\), \(T \in \{T_\alpha\}\), and thus it follows from the definition of set intersection that \(\cap T_\alpha \subseteq T\). Therefore, the double containment of \(\cap T_\alpha\) and \(T\) imply that \(\cap T_\alpha = T\). \newline \\
Now, suppose that \(A\) is a subbasis, meaning topology \(T\) on \(X\) is generated by the subbasis \(A\). Then, by the same reasoning as the second paragraph of the solution, we still have that \(\cap T_\alpha \subseteq T\). \newline \\
To demonstrate that \(T \subseteq \cap T_\alpha\), let \(S \in T\) be arbitrary. By the definition of subbasis, it follows that \(S = \cup(\cap_{x \in A'} x)\) with \(A' \subseteq A\), , where the outer union is an arbitrary union and the inner intersection is a finite intersection. Notice that the inner intersection is a subset of \(A\), and since \(A \in \cap T_\alpha\) as the elements of the intersection contain \(A\), by the definition of topology on \(X\), it follows that the finite intersection is contained as well, i.e. \(\cap_{x \in A'} x \in \cap T_\alpha\). Since arbitrary union is closed by the definition of topology, it follows that \(S = \cup(\cap_{x \in A'} x) \in \cap T_\alpha\). Since \(S \in T\) was arbitrary, it follows that \(T \subseteq \cap T_\alpha\), and by double containment of \(T\) and \(\cap T_\alpha\), it also follows that \(T = \cap T_\alpha\), similar as when \(A\) is a basis. 
\end{proof}

\newpage

\begin{problem}{13.6} \ \\
Show that the topologies of \(\mathbb{R}_l\) and \(\mathbb{R}_k\) are not comparable.
\end{problem}

\begin{proof}
Let \(T_l\) be the topology of \(\mathbb{R}_l\) and let \(T_k\) be the topology of \(\mathbb{R}_k\). Consider \([0, b) \in T_l\) with \(0 < b \in \mathbb{R}\). Notice that there is no element in \((0, b) - \{\frac{1}{n} : n \in \mathbb{Z}_+\}\) that contains \(0\) and is a subset of \([0, b)\), thus it follows that \(T_l \not \subseteq T_k\). \newline \\
Now, let I = \((-1, 1) - \{\frac{1}{n} : n \in \mathbb{Z}_+\} \in T_k\). Notice that \(0 \in I\), but there does not exist an element in \(T_l\) that contains \(0\) and is of form \([a, b)\) that is a subset of \(T_l\) ["holes" in the intervals prevent such inclusion]. Hence, it follows that \(T_k \not \subseteq T_l\), and thus \(T_l\) and \(T_k\) are not comparable. 
\end{proof}

\end{document}