\documentclass[12pt]{article}
\usepackage[margin=1in]{geometry}

\usepackage{latexsym, amssymb, amsmath, amsfonts, amscd, amsthm}
\usepackage{enumerate, hyperref, multicol, tikz}

\theoremstyle{definition}
\newtheorem{definition}{Definition}
\newtheorem{example}{Example}
\newtheorem{exercise}{Exercise}
\newtheorem{question}{Question}
\newtheorem*{notation*}{Notation}

\theoremstyle{theorem}
\newtheorem{fact}{Fact}
\newtheorem{lemma}{Lemma}
\newtheorem{proposition}{Proposition}
\newtheorem{theorem}{Theorem}
\newtheorem{corollary}{Corollary}

\newcommand{\NN}{\mathbb{N}}
\newcommand{\ZZ}{\mathbb{Z}}
\newcommand{\QQ}{\mathbb{Q}}
\newcommand{\RR}{\mathbb{R}}
\newcommand{\CC}{\mathbb{C}}
\newcommand{\FF}{\mathbb{F}}


\begin{document}
\section{Definitions}
\begin{definition}\label{def: simple connected graph}\cite{edelsbrunner2010computational}
	A simple graph is \emph{connected} if there is a path between every pair of vertices.
\end{definition}

\begin{definition}\label{def: topology}\cite{munkres2000topology}
	A \emph{topology} on a set \(X\) is a collection \(\mathcal{T}\) of subsets of \(X\) having the following properties:
	\begin{enumerate}[(a)]
		\item \(\emptyset\) and \(X\) are in \(\mathcal{T}\).
		\item The union of any subcollection of \(\mathcal{T}\) is in \(\mathcal{T}\).
		\item The intersection of the elements of any finite subcollection of \(\mathcal{T}\) is in \(\mathcal{T}\).
	\end{enumerate}
\end{definition}

\begin{definition}\label{def: finer coarser comparable}\cite{munkres2000topology}
	Suppose that $\mathcal{T}$ and $\mathcal{T}'$ are two topologies on a given set $X$. If $\mathcal{T}' \supset \mathcal{T}$, we say that $\mathcal{T}'$ is finer than $\mathcal{T}$; if $\mathcal{T}' \it{properly}$ contains $\mathcal{T}$, we say that $\mathcal{T}'$ is strictly finer than $\mathcal{T}$. We also say that $\mathcal{T}$ is coarser than $\mathcal{T}'$, or strictly coarser, in these two respective situations. We say $\mathcal{T}$ is comparable with $\mathcal{T}'$ if either $\mathcal{T}' \supset \mathcal{T}$ or $\mathcal{T} \supset \mathcal{T}'$.
\end{definition}

\begin{definition}\label{def: basis for a topology}\cite{munkres2000topology}
If \(X\) is a set, a \emph{basis} for a topology on \(X\) is a collection of $\mathit{B}$ of subsets of \(X\) (called \emph{basis elements} such that 
\begin{enumerate}[(1)]
\item For each $x \in X$, there is at least one basis element $B$ containing $x$. 
\item If $x$ belongs to the intersection of two  basis elements $B_1$ and $B_2$, then there is a basis element $B_3$ containing $x$ such that $B_3 \subset B_1 \cap B_2$
\end{enumerate}
\end{definition} 

\begin{definition}\label{def: subbasis}\cite{munkres2000topology}
A subbasis \(S\) for a topology \(X\) is a collection of subsets of \(X\) whose union equals \(X\). The topology generated by the subbasis of \(S\) is defined to be the collection \(T\) of all unions of finite intersection s of elements of \(S\). 

\begin{definition}\label{def: subspace topology}\cite{munkres2000topology}
Let \(X\) be a topological space with topology \(T\). If \(Y\) is a subset of \(X\), the collection $T_y = \lbrace{Y cap U | U \in T \rbrace}$ is a topology on \(Y\), called the \emph{subspace topology}. With this topology, \(Y\) is called a \emph{subspace\} of \(X\); its open sets consist of all intersections of open sets of \(X\) with \(Y\)$. 


\end{definition}

\section{Theorems}
\begin{theorem}[Jordan Curve Theorem]\cite{?}
	\label{thm: Jordan Curve Theorem}
	Removing the image of a simple closed curve from $\mathbb{R}^2$ leaves two connected components, the bounded \textbf{inside} and the un-bounded \textbf{outside}. The inside together with the image of the curve is homeo-morphic to a closed disk.
\end{theorem}

\begin{theorem}[Sch{\"o}nflies Theorem]\cite{wolframSchönflies}
	\label{thm: Schonflies Theorem}
	If J is a simple closed curve in $\mathbb{R}^2$, the closure of one of the compononents of $\mathbb{R}^2 - J$ is homeomorphic with the unit 2-ball. 
\end{theorem}

\begin{theorem}[Kuratowski Theorem]\cite{edelsbrunner2010computational}
	A simple graph is planar iff no subgraph is home-omorphic to $K_5$ or to $K_{3,3}$.
\end{theorem}

\begin{theorem}[Tutte’s Theorem]\cite{edelsbrunner2010computational}
	Let G = (V, E) be the edge-skeleton of a triangulation of the disk and $f : V \to \mathbb{R}^2$ a strictly convex combination mapping that maps the boundary vertices to the corners of a strictly convex polygon. Then drawing straight edges between the image points gives a straight-line embedding.
\end{theorem}

\begin{theorem}[Geometric Realization Theorem]\cite{edelsbrunner2010computational}
	 Every abstract simplicial complex of dimension d has a geometric realization in $\mathbb{R}^{2d+1}$.
\end{theorem}

\begin{theorem}[Simplicial Approximation Theorem]\cite{edelsbrunner2010computational}
	If g: $|K| \to |L|$ is continuous then there is a sufficiently large integer n such that g has simplicial approximation f: $Sd^n K \to L$. 
\end{theorem}

\begin{theorem}[Helley's Theorem]\cite{edelsbrunner2010computational}
	Let F be a finite collection of closed, convex sets in $\mathbb{R}^n$. Every $d+1$ of the sets have a non-empty common intersection iff they all have a non-empty common intersection. 
\end{theorem}

\begin{theorem}[Nerve Theorem]\cite{edelsbrunner2010computational}
	Let F be a finite collection of closed, convex sets in Euclidean space. Then the nerve of F and the union of the sets in F have the same homotopy type.
\end{theorem}

\begin{theorem}[Brouwer's Fixed Point Theorem]\cite{edelsbrunner2010computational}
	A continuous map f: $\mathbb{B}^{p+1} \to \mathbb{B}^{p+1}$ has at least one fixed point $x = f(x)$.
\end{theorem}

\begin{theorem}[Euler-Poincar{\'e} Theorem]\cite{edelsbrunner2010computational}
	The Euler characteristic of a topological space is the alternating sum of its Betti numbers, $X = \sum_{p\geq0} (-1)^p \beta_p$
\end{theorem}

\begin{theorem}[Excision Theorem]\cite{edelsbrunner2010computational}
	Let $K_0\subseteq K$ and $L_0 \subseteq L$ be pairs of simplicial complexes that satisfy $L\subseteq K$ and $L - L_0 = K- K_0$. Then they have isomorphic relative homology groups, that is, $H_p (K,K_0) \simeq H_p (L,L_0)$ for all dimensions p. 
\end{theorem}

\begin{theorem}[Exact Sequence of a Pair Theorem]\cite{edelsbrunner2010computational}
	Let K be a simplicial complex and $K_0 \subseteq K$ be a subcomplex. Then there is a long exact sequence
	\begin{align*} 
	... \to H_p (K_0) \to H_p (K) \to H_p (K,K_0) \to H_{p-1} (K_0) \to ...
	\end{align*}
\end{theorem}

\bibliographystyle{plain}
\bibliography{sources}

\end{document}
